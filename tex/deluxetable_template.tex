\begin{longrotatetable}
\movetabledown=0.25in
\newcolumntype{z}{>{\raggedright\arraybackslash}p{1.5in}}
\begin{deluxetable*}{lcccccccccz}
\tablecaption{Targets surveyed by \galex with epochs before and after discovery 
and $z<0.5$. The complete table will be available as supplementary material.
\label{tab:Targets}}
\tabletypesize{\footnotesize}
\tablehead{
    \colhead{Target} & \colhead{Disc. Date} & \colhead{\galex Coordinates} & 
    \colhead{Obs.} & \colhead{$\Delta t_\text{first}$} & 
    \colhead{$\Delta t_\text{last}$} & \colhead{$\Delta t_\text{next}$} & 
    \colhead{Redshift} & \colhead{Distance} & \colhead{$A_V$} & \colhead{Reference(s)} \\
    \colhead{Name} & \colhead{} & \colhead{$\alpha$, $\delta$ (J2000)} & 
    \colhead{Epochs} & \colhead{[days]} & 
    \colhead{[days]} & \colhead{[days]} & 
    \colhead{} & \colhead{[Mpc]} & \colhead{[mag]} & \colhead{}
}
\startdata
===
\enddata
\tablenotetext{a}{Offset between NED entry and \galex sky coordinates is larger
    than 30 kpc}
\tablenotetext{b}{Redshift-independent distance from \citet{Tully2016-Cosmicflows3}}
\tablecomments{Discovery dates were retrieved from the Open Supernova Catalog; 
redshift, Hubble distance, extinction, and host morphology were retrieved from NED. Columns 5-7 list the number of days between the first \galex observation and SN discovery ($\Delta t_\text{first}$), the number of days between discovery and the last \galex observation ($\Delta t_\text{last}$), and the number of days between discovery and the next \galex observation ($\Delta t_\text{next}$)}
% TODO find a better way to present these definitions
\end{deluxetable*}
\end{longrotatetable}