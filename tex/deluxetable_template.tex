\begin{longrotatetable}
\movetabledown=0.5in
\begin{deluxetable*}{lcccccccccccll}
\tablewidth{700pt}
\tablecaption{Targets surveyed by \galex with epochs before and after discovery 
and $z<0.5$.\label{tab:Targets}}
\tabletypesize{\scriptsize}
\movetabledown=0.5in
\tablehead{
    \multicolumn1l{Target} & \colhead{Disc. Date} & \colhead{\galex Coordinates} & 
    \colhead{Obs.} & \colhead{$t_\text{disc} - t_\text{first}$} & 
    \colhead{$t_\text{last} - t_\text{disc}$} & \colhead{$\Delta t_\text{nearest}$} & 
    \colhead{Redshift} & \colhead{Distance} & \colhead{$A_V$} & 
    \colhead{Host} & \multicolumn1l{Reference(s)} \\
    \multicolumn1l{Name} & \colhead{} & \colhead{$\alpha$, $\delta$ (J2000)} & 
    \colhead{Epochs} & \colhead{[days]} & 
    \colhead{[days]} & \colhead{[days]} & 
    \colhead{} & \colhead{[Mpc]} & \colhead{[mag]} & 
    \colhead{Morph.} & \colhead{}
}
\startdata
===
\enddata
\tablecomments{Discovery dates were retrieved from the Open Supernova Catalog; redshift, Hubble 
distance, extinction, and host morphology were retrieved from NED. The complete 
table will be available as supplementary material.}
\end{deluxetable*}
\end{longrotatetable}